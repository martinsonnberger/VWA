\chapter{Einleitung}

Autonomes Fahren, ein Thema, das unsere Zukunft massiv verändern wird. Vermutlich wird man in Zukunft kein eigenes Auto mehr benötigen, um täglich in die Arbeit zu kommen. Stattdessen wird man von einem selbstfahrenden Taxi abgeholt, dass jeden Tag um 7 Uhr vor der Haustür steht, um einen abzuholen.

Es ist wichtig, vor der neuen Technologie nicht zurückzuschrecken, sondern ihr eine Chance zu geben, unser Leben einfacher und vor allem sicherer zu gestalten. Elon Musk sagte: \zit{elon-musk-quote}{When Henry Ford made cheap, reliable cars people said, \enq{Nah, what's wrong with a horse?} That was a huge bet he made, and it worked.}

\bigskip

Das Ziel dieser Arbeit ist es, die Technologie, die hinter autonomen Fahrzeugen steckt, zu erläutern, um die Funktionsweise der Autos, denen wir künftig unser Leben anvertrauen werden, ein wenig besser zu verstehen und zu durchblicken.

Der erste Teil dieser Literaturarbeit liefert einen Überblick über die allgemeine Funktionsweise autonomer Fahrzeuge. Es werden die verschiedenen Sensoren und Computersysteme, die im Fahrzeug verbaut sind, dargelegt. Außerdem werden die verschiedenen Autonomiestufen erklärt, die Fahrzeuge in ihren Grad der Automatisierung einteilen.

Im nächsten Kapitel wird besonders auf die Positionsbestimmung autonomer Fahrzeuge eingegangen, indem die verschiedenen Methoden der Lokalisierung genauer erläutert werden. Dabei wird sowohl das weit verbreitete \ac{GPS}, als auch \acs{Radar} und \acs{Lidar} genauer erläutert.

Das letzte Kapitel behandelt den aktuellen Stand der Dinge und erklärt, was heute bereits mit autonomen Fahrzeugen möglich ist, indem verschiedenste Projekte zum autonomen Fahren vorgestellt werden. Außerdem beschäftigt sich das Kapitel mit aktuellen Systemen, die bei der Positionsbestimmung ihren Einsatz finden.
