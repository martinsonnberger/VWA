\chapter{Einleitung}

Autonomes Fahren, ein Thema, das unsere Zukunft massiv verändern wird. Benötigt man noch ein eigenes Auto, um zur Arbeit zu fahren? Oder bestellt man einfach ein autonomes Taxi über eine App? Oder noch besser, kommt das autonome Taxi täglich um 7 Uhr vor die Haustür, weil es weiß, dass man um diese Uhrzeit zur Arbeit fahren muss? Das alles sind Fragen, die uns in einigen Jahren sicherlich beschäftigen werden, sich heute noch nicht genau beantworten lassen können.

Es ist wichtig, vor der neuen Technologie nicht zurückzuschrecken, sondern ihr eine Chance zu geben, unser Leben einfacher und vor allem sicherer zu gestalten. Elon Musk sagte: \zit{elon-musk-quote}{When Henry Ford made cheap, reliable cars people said, \enq{Nah, what's wrong with a horse?} That was a huge bet he made, and it worked.}

\bigskip

Diese Vorwissenschaftliche Arbeit soll die Technologie, die hinter autonomen Fahrzeugen steckt, erläutern, um die Funktionsweise der Autos, denen wir künftig unser Leben anvertrauen werden, ein wenig besser zu verstehen und zu durchblicken.

Das erste Kapitel dieser Literaturarbeit liefert einen Überblick über die allgemeine Funktionsweise autonomer Fahrzeuge. Außerdem werden die verschiedenen Autonomiestufen erklärt, die Fahrzeuge in ihren Grad der Automation einteilen.

Im zweiten Kapitel wird besonders auf die Positionsbestimmung autonomer Fahrzeuge eingegangen, indem die verschiedenen Methoden der Lokalisierung genauer erläutert werden.

Das dritte Kapitel behandelt den aktuellen Stand der Dinge und erklärt, was heute bereits mit autonomen Fahrzeugen möglich ist. Außerdem wird versucht, einen Blick in die Zukunft zu werfen und eine Schätzung abzugeben wann wir in einer Welt voller autonomer Fahrzeuge leben werden.
