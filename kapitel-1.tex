\chapter{Autonomes Fahren --- Beschreibung allgemein}
\section{Begriffsdefintion}
Um den Begriff \enq{Autonomes Fahren} definieren zu können, muss in der Zeit ein Sprung zurück gemacht werden, nämlich zur Erfindung des Automobils im Jahr 1886. Der Begriff \enq{Automobil} setzt sich aus dem griechischen \textit{autòs} (\enq{selbst}) und dem lateinischen \textit{mobilis} (\enq{beweglich}) zusammen, bedeutet also \enq{Selbstbewegliche}. Damals sollte dadurch, die von nun an bestehende Unabhängigkeit von Pferden verdeutlicht werden. Genauer betrachtet fällt auf, dass durch das Wegfallen von Kutschpferden die Autonomie in gewisser Weise verloren ging. Durch jahrelanges Training konnten sich die Pferde in Maßen selbst fortbewegen, indem sie beispielsweise einen nicht mehr fahrtüchtigen Kutscher alleine nachhause brachten.

Durch die Einführung von zahlreichen Fahrerassistenzsystemen und letzendlich des autonomen Fahrens wird dem Automobil seine ursprünglich gedachte Autonomität nicht nur wieder zurückgegeben, sondern auch um ein Vielfaches erweitert.

\zit{Definition}{Autonomes Fahren bedeutet das selbständige, zielgerichtete Fahren eines Fahrzeugs im realen Verkehr, ohne Eingriff des Fahrers.}

\section{Automatisierungsgrade}
Der Grad der Automatisierung kann in verschiedene Stufen eingeteilt werden.
