\chapter{Autonomes Fahren --- Beschreibung allgemein}
\section{Begriffsdefintion}
Um den Begriff \enq{Autonomes Fahren} zu definieren, muss in der Zeit ein Sprung zurück gemacht werden, nämlich zur Erfindung des Automobils im Jahr 1886. Der Begriff \enq{Automobil} setzt sich aus dem griechischen \textit{autòs} (\enq{selbst}) und dem lateinischen \textit{mobilis} (\enq{beweglich}) zusammen, bedeutet also \enq{Selbstbewegliche}. Damals sollte dadurch, die von nun an bestehende Unabhängigkeit von Pferden verdeutlicht werden. Genauer betrachtet fällt auf, dass durch das Wegfallen von Kutschpferden die Autonomie in gewisser Weise verloren ging. Durch jahrelanges Training konnten sich die Pferde in Maßen selbst fortbewegen, indem sie beispielsweise einen nicht mehr fahrtüchtigen Kutscher alleine nachhause brachten.
\vgl[2-3]{Maurer}
Durch die Einführung von zahlreichen Fahrerassistenzsystemen und letzendlich des autonomen Fahrens wird dem Automobil seine ursprünglich gedachte Autonomie nicht nur wieder zurückgegeben, sondern auch um ein Vielfaches erweitert.

Autonomes Fahren beschreibt die Übergabe aller mit der Fahrzeugführung verbundenen Tätigkeiten wie beschleunigen, bremsen, lenken, etc., an eine Maschine, den sogenannten Fahrroboter, welcher selbstständig, ohne Mitwirken des Menschen, Entscheidungen trifft. Dabei darf von autonomen Fahrzeugen keine größere Gefahr, als von Menschen gesteuerten ausgehen. Primäre Informationsquellen sind visuelle Quellen, wie sie auch dem Fahrer zur Verfügung stehen.
\vgls[667]{Winner}{Definition}

\dzit{Definition}{Autonomes Fahren bedeutet das selbständige, zielgerichtete Fahren eines Fahrzeugs im realen Verkehr, ohne Eingriff des Fahrers.}

\section{Autonomiestufen}
Der Grad der Automatisierung kann in verschiedene Stufen eingeteilt werden, welche Autonomiestufen oder Level genannt werden. Die Einteilung erfolgt in sechs Stufen, welche in Europa u. a. von der deutschen Bundesanstalt für Straßenwesen und in den USA von der SAE International festgelegt werden.

%needs fix: Quelle nur für Tabelle
\zit{wiki-levels}{}

\begin{table}[p]
    \begin{tabularx}{\textwidth}{P{1.2cm}lX}
      \toprule

      \textbf{SAE Level} & \textbf{Name} & \textbf{Definition} \\

      \midrule
      \multicolumn{3}{l}{Menschlicher Fahrer kontrolliert die Umgebung.} \\
      \midrule

      0 & Keine Automation & die durchgängige Ausführung aller Aspekte der dynamischen Fahraufgabe durch den menschlichen Fahrer, auch wenn unterstützende Warn- oder Interventionssysteme eingesetzt werden. \\[0.3cm]

      1 & Fahrerassistenz & eine Fahrer-Assistenz, die Fahrmodus-spezifische Aufgaben wie die Lenk-Assistenz oder Beschleunigungs- /Brems-Assistenz dank der Verwendung von Fahr- und Umgebungsinformationen ausführt und mit der Erwartung, dass der menschliche Fahrer alle verbleibenden Aspekte der dynamischen Fahraufgabe ausführt \\[0.3cm]

      2 & Teilautomation & die Fahrmodus-spezifische Ausführung von Lenk- und Beschleunigungs- /Bremsvorgängen durch ein oder mehrere Fahrerassistenzsysteme unter Verwendung von Informationen über die Fahrumgebung und mit der Erwartung, dass der menschliche Fahrer alle verbleibenden Aspekte der dynamischen Fahraufgabe ausführt \\

      \midrule
      \multicolumn{3}{l}{Das System kontrolliert die Umgebung.} \\
      \midrule

      3 & Bedingte Automation & die Fahrmodus-spezifische Ausführung aller Aspekte der dynamischen Fahraufgabe durch ein automatisiertes Fahrsystem mit der Erwartung, dass der menschliche Fahrer auf Anfrage des Systems angemessen reagieren wird \\[0.3cm]

      4 & Hohe Automation & die Fahrmodus-spezifische Ausführung aller Aspekte der dynamischen Fahraufgabe durch ein automatisiertes Fahrsystem, selbst wenn der menschliche Fahrer auf Anfrage des Systems nicht angemessen reagiert \\[0.3cm]

      5 & Volle Automation & die durchgängige Ausführung aller Aspekte der dynamischen Fahraufgabe durch ein automatisiertes Fahrsystem unter allen Fahr- und Umweltbedingungen, die von einem menschlichen Fahrer bewältigt werden können \\

      \bottomrule
    \end{tabularx}
  \caption[Einteilung der Autonomiestufen nach SAE J3016. Quelle: \fullcite{sae-j3016}.]{Einteilung der Autonomiestufen nach SAE J3016 (\quotecite(vgl.)()[19]{sae-j3016}[]{wiki-levels})}
  \label{levels}
\end{table}

