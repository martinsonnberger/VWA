\chapter{Autonomes Fahren --- Beschreibung allgemein}

Versucht man den Begriff \enq{Autonomes Fahren} zu definieren, stößt man unweigerlich auf das Automobil und dessen Erfindung im Jahr 1886. Der Begriff \enq{Automobil} setzt sich aus dem griechischen \textit{autòs} (\enq{selbst}) und dem lateinischen \textit{mobilis} (\enq{beweglich}) zusammen, bedeutet also \enq{Selbstbewegliche}. Damals sollte dadurch die von nun an bestehende Unabhängigkeit von Kutschpferden verdeutlicht werden. Genauer betrachtet fällt auf, dass durch das Wegfallen von Pferden die Autonomie in gewisser Weise verloren ging. Durch jahrelanges Training konnten sich die Pferde in Maßen selbst fortbewegen, indem sie beispielsweise einen nicht mehr fahrtüchtigen Kutscher alleine nachhause brachten.
\vgl[2-3]{Maurer}
Durch die Einführung von zahlreichen Fahrerassistenzsystemen und letzendlich des autonomen Fahrens wird dem Automobil seine ursprünglich gedachte Autonomie nicht nur wieder zurückgegeben, sondern auch um ein Vielfaches erweitert.


\section{Begriffsdefintion}

\dzit{Definition}{Autonomes Fahren bedeutet das selbständige, zielgerichtete Fahren eines Fahrzeugs im realen Verkehr, ohne Eingriff des Fahrers.}

Autonomes Fahren beschreibt die Übergabe aller mit der Fahrzeugführung verbundenen Tätigkeiten wie beschleunigen, bremsen, lenken, etc., an eine Maschine, den sogenannten Fahrroboter, welcher selbstständig, ohne Mitwirken des Menschen, Entscheidungen trifft. Dabei darf von autonomen Fahrzeugen keine größere Gefahr, als von Menschen gesteuerten ausgehen. Primäre Informationsquellen sind visuelle Quellen, wie sie auch dem Fahrer zur Verfügung stehen.
\vgls[667]{Winner}{Definition}

Ein Punkt, in dem sich autonom gesteurte Fahrzeuge von heutigen Maschinen unterscheiden, ist die Unvorhersehbarkeit ersterer. Maschinen im heutigen Alltag werden entweder von Menschen gesteuert (\zB Autos, Baumaschinen), oder führen stark eingeschränkte, sich wiederholende Bewegungen durch (\zB Aufzüge, Rolltreppen). Autonome Fahrzeuge hingegen bewegen sich auf normalen Straßen und fahren auf beinahe allen Wegen, die auch ein menschlicher Fahrer benutzen würde. Zusammen mit der alleinig durch den Computer stattfindenden Entscheidungsfindung, lässt sie das in ihren Entscheidungen unvorhersehbar machen.

Selbstfahrende Autos stellen eine neue Klasse an von Computern kontrollierten Systemen dar, in der die Maschine Entscheidungen für sich selbst trifft. Um solche Systeme zu verstehen, ist es umso wichtiger, ein Verständnis der dahintersteckenden Technologie von autonomen Fahrzeugen zu haben. Einer Technologie, die es ihnen erlaubt, sich selbst durch das komplexe Straßennetz zu manövrieren.
\vgl[130]{Surden}

\subsection{Autonomiestufen}
Der Grad der Automatisierung eines Fahrzeugs kann in verschiedene Stufen eingeteilt werden, welche Autonomiestufen oder Level genannt werden. Eine erste Einteilung in fünf Stufen erfolgte 2013 in den USA von der \ac{NHTSA}. Im Jahr 2014 veröffentlichete die \ac{SAE} zum ersten Mal den \textit{J3016} Standard, welcher eine Überarbeitung der ursprünglichen Einteilung der \ac{NHTSA} darstellt. Der Hauptgrund für die Adaptierung war die Standardisierung verschiedener Skalen. Der größte Unterschied besteht in einer zusätzlichen, sechsten Stufe, die für die Einteilung verwendet wird. \vgl[11-13]{Kaufleitner} Der \textit{J3016} Standard ist die heute am weitesten verbreitete Einteilung, auch das Bundesministerium für Verkehr, Innovation und Technologie verwendet ihn als Grundlage. (H. Atasayar, private Kommunikation, 20. Aug. 2018)
%TODO: fragen wegen privater kommunikation

Am Beginn der Skala steht Level 0, was bedeutet, dass sämtliche Fahraufgaben durch den Menschen erfolgen. Level 5 hingegen steht für eine vollständige Automation, bei der jegliche, für eine Fahrt erforderlichen Aufgaben, durch den Computer ausgeführt werden. Bei Begriffen wie \enq{selbstfahrend}, \enq{fahrerlos} oder auch \enq{autonom}, ist meist von einer vollständigen Automation auf Level 5 die Rede.

In der Zusammenfassung der Autonomiestufen (\ref{levels}) ist oft von der \textit{Dynamischen Fahraufgabe} die Rede, die \ac{SAE} definiert sie folgendermaßen: \zit[6]{sae-j3016}{All of the real-time operational and tactical functions required to operate a vehicle in on-road traffic, excluding the strategic functions such as trip scheduling and selection of destinations and waypoints \elp{}}

%TODO: Fragen ob das stimmt
\nocite{wiki-levels}
\begin{table}[p]
    \begin{tabularx}{\textwidth}{P{1.2cm}lX}
      \toprule

      \textbf{SAE Level} & \textbf{Name} & \textbf{Definition} \\

      \midrule
      \multicolumn{3}{l}{Menschlicher Fahrer kontrolliert die Umgebung.} \\
      \midrule

      0 & Keine Automation & die durchgängige Ausführung aller Aspekte der dynamischen Fahraufgabe durch den menschlichen Fahrer, auch wenn unterstützende Warn- oder Interventionssysteme eingesetzt werden. \\[0.3cm]

      1 & Fahrerassistenz & eine Fahrer-Assistenz, die Fahrmodus-spezifische Aufgaben wie die Lenk-Assistenz oder Beschleunigungs- /Brems-Assistenz dank der Verwendung von Fahr- und Umgebungsinformationen ausführt und mit der Erwartung, dass der menschliche Fahrer alle verbleibenden Aspekte der dynamischen Fahraufgabe ausführt \\[0.3cm]

      2 & Teilautomation & die Fahrmodus-spezifische Ausführung von Lenk- und Beschleunigungs- /Bremsvorgängen durch ein oder mehrere Fahrerassistenzsysteme unter Verwendung von Informationen über die Fahrumgebung und mit der Erwartung, dass der menschliche Fahrer alle verbleibenden Aspekte der dynamischen Fahraufgabe ausführt \\

      \midrule
      \multicolumn{3}{l}{Das System kontrolliert die Umgebung.} \\
      \midrule

      3 & Bedingte Automation & die Fahrmodus-spezifische Ausführung aller Aspekte der dynamischen Fahraufgabe durch ein automatisiertes Fahrsystem mit der Erwartung, dass der menschliche Fahrer auf Anfrage des Systems angemessen reagieren wird \\[0.3cm]

      4 & Hohe Automation & die Fahrmodus-spezifische Ausführung aller Aspekte der dynamischen Fahraufgabe durch ein automatisiertes Fahrsystem, selbst wenn der menschliche Fahrer auf Anfrage des Systems nicht angemessen reagiert \\[0.3cm]

      5 & Volle Automation & die durchgängige Ausführung aller Aspekte der dynamischen Fahraufgabe durch ein automatisiertes Fahrsystem unter allen Fahr- und Umweltbedingungen, die von einem menschlichen Fahrer bewältigt werden können \\

      \bottomrule
    \end{tabularx}
  \caption[Einteilung der Autonomiestufen nach SAE J3016. Quelle: \fullcite{sae-j3016}.]{Einteilung der Autonomiestufen nach SAE J3016 (\quotecite(vgl.)()[19]{sae-j3016}[]{wiki-levels})}
  \label{levels}
\end{table}



\section{Technologie von autonomen Fahrzeugen}

%TODO: Sekundärzitat: Quelle finden!
Bei einem hohen Automatisierungsgrad wird die Technologie verwendet, um die folgenden drei primären Fragen zu beantworten: \vgl[137]{Surden}
\begin{enumerate}
  \item{Wo befindet sich das Fahrzeug?}
  \item{Welche Objekte befinden sich um das Fahrzeug herum?}
  \item{Wo ist es gewünscht, erlaubt und sicher, sich als nächstes hinzubewegen?}
\end{enumerate}
Einem autonomen Fahrzeug muss es möglich sein, diese Fragen unter Berücksichtung von Verkehrszeichen, Ampeln und Straßenmarkierungen, zu beantworten. Außerdem muss es in einem Raum voll mit Objekten wie Fahrzeugen, Fußgängern, Radfahrern und Tieren, sicher navigieren können. Um diese Aufgaben erfolgreich zu bewältigen, ist ein Zusammenspiel von \textbf{Hardware} und \textbf{Software} notwendig.

\subsection{Sensoren}
In autonomen Fahrzeugen sind zahlreiche Sensoren installiert, die dazu beitragen, alle drei oben genannten Fragen, zu beantworten. Sensoren sind Komponenten, die eine physikalische Größe der Umgebung oder des Fahrzeuges selbst in ein elektrisches Signal umwandeln, welches an den Fahrzeugcomputer weitergeleitet wird, um es dort weiterzuverarbeiten. \vgl{sensor-def} In autonomen Fahrzeugen wird meist ein \acs{Radar} (\acl{Radar}) verwendet, um Hindernisse zu erkennen, sowie deren Entfernung und Geschwindigkeit zu bestimmen. Dazu werden elektromagentische Wellen im Frequenzbereich der Mikrowellen (meist \SI{24}{\giga\hertz}) ausgesendet. Treffen sie auf ein Objekt, werden die Wellen reflektiert und vom Empfänger des \acs{Radar}, welcher meist gleichzeitig auch den Sender darstellt, aufgefangen. Mithilfe des gemessenen Zeitunterschiedes können nun Entfernung und Geschwindigkeit des Objekts berechnet werden. \vgl{sensors-thedrive}

Ein weiterer, bereits in heutigen, konventionellen Fahrzeugen eingesetzter Sensor, ist der Ultraschallsensor. Aufgrund seiner Beschränkungen in Hinsicht auf Reichweite und Maximalgeschwindigkeit ist er jedoch nicht so vielseitig anwendbar wie ein \acs{Radar}. Wegen seiner Kompaktheit sowie kostengünstigen Herstellung ist er jedoch besonders bei Einparksystemen beliebt.

Mithilfe von \acs{Lidar}-Scannern (\acl{Lidar}), ist es möglich, nicht nur Objekte an sich, sondern auch um welche Art von Objekt es sich handelt (anderes Fahrzeug, Fußgänger, Ampel, etc.), zu erkennen. Hochleistungs-\acs{Lidar}-Systeme können sogar erkennen, in welche Richtung ein Fußgänger gerade geht und aufgrund dessen seinen weiteren Weg schätzen. \vgl{waymo} Bei einem \acs{Lidar} werden schwache, unsichtbare Laserstrahlen anstatt Mikrowellen ausgesendet, um auf dem selben Prinzip wie bei \acs{Radar} und Ultraschall Hindernisse zu indentifizieren. \acs{Lidar}-Scanner werden auch zur Positionsbestimmung autonomer Fahrzeuge verwendet. (Siehe \ref{kapitel-2})

\subsection{Annotierte digitale Karten}\label{maps}

Autonome Fahrzeuge sind neben Sensoren auch mit Kartenmaterial ausgestattet, welches speziell für autonomes Fahren im Vorhinein konstruiert wird. Einerseits sind diese Karten mit geographischen Informationen zu Straßenlayout und markanten Punkten (Längen- und Breitengrade), andererseits mit zahlreichen Zusatzinformationen ausgestatten, die insbesondere für autonomes Fahren benötigt werden. \vgl[138]{Surden}

Annotierte digitale Karten beinhalten ein detailliertes Bild der Straßen, welches meist durch einen dreidimensionalen Laser-Scan hergestellt wird. Dieser Scan wird mit speziellen Vermessungsfahzeugen durchgeführt, die das Straßennetz abfahren, in dem sich autonome Fahrzeuge später aufhalten werden.
Weiters beinhalten sie manuell hinzugefügte Information über lokale Gegebenheiten wie Ampelanlagen, Hauseinfahrten oder Straßenmarkierungen. \vgl[139]{Surden}

Durch diese Art von Kartenmaterial wird autonomes Fahren deutlich effizienter, da Fahrzeuge sich bereits im Vorhinein auf Situationen einstellen können und dementsprechend rechtzeitig reagieren. Durch den Vergleich mit Echtzeitinformation von Kameras und \acs{Lidar}-Sensoren am Fahrzeug, wird die Fähigkeit, Straßenschilder oder Ampeln zu erkennen, deutlich erhöht. \vgl[140]{Surden}

Sollte der Fall eintreten, dass das vorab gespeicherte Kartenmaterial nicht mit den Sensordaten übereinstimmt, können autonome Fahrzeuge auch allein mit Sensoren sicher und akurat durch den Verkehr steuern. Außerdem ist es möglich, das Kartenmaterial dynamisch den veränderten Gegebenheiten anzupassen. Wird beispielsweise an einer Kreuzung eine neue Ampelanlage installiert und 100 Fahrzeuge melden dies einem zentralen Kartenserver, so kann die Karte automatisch angepasst werden, um wieder dem aktuellen Stand zu entsprechen. \vgl[140]{Surden}

\subsection{Koordinierendes Computersystem}

Das koordinierende Computersystem ist neben den Sensoren und annotierten digitalen Karten die dritte ausschlagebende Technologie, um autonomes Fahren zu ermöglichen. Das System kombiniert sämtliche Sensordaten und vergleicht sie mit den Kartendaten, um feststellen zu können, ob es sicher, erwünscht und legal ist, das Fahrzeug an eine neue Position zu bewegen. Trifft dies zu, steuert das System das Fahrzeug zu dieser Position.
\vgl[141]{Surden}


\section{Prozess des autonomen Fahrens}

Wie in vielen Anwendungsfällen in der Robotik, wird auch beim autonomen Fahren der dreistufige \textit{sense-plan-act}-Prozess verwendet. \vgl{Anderson} In der ersten Stufe, der \textit{Erkennungsphase}, werden sämtliche Sensoren am Fahrzeug verwendet, um festzustellen, wo sich das Fahrzeug befindet und von welchen Hindernissen es umgeben ist. In der \textit{Planungsphase} werden diese Informationen an das koordinierende Computersystem weitergeleitet, welches die Daten analysiert. Der Computer erstellt ein Modell der Umgebung inklusive aller Objekte die das Fahrzeug umgeben. Komplexe Algorithmen entscheiden nun anhand von eingegebenem Ziel, Straßenmarkierungen, Verkehrszeichen und Hindernissen, wo sich das Fahrzeug als nächstes hinbewegen soll. In der \textit{Ausführungsphase} steuert der Computer schließlich das Fahrzeug, indem die Fahrzeugtechnik (Beschleunigung, Bremsen, Lenkung) elektronisch angesteuert wird.
\vgl[141]{Surden}

Zudem läuft dieser Prozess in einem ständigen Kreislauf ab, welcher sich mehrere tausend Male pro Sekunde wiederholt, was dazu führt, dass ein autonomes Fahrzeug beinahe ohne Verzögerung auf Veränderungen in der Umwelt reagieren kann. Ein Fahrzeug hat beispielsweise entschieden, dass es sicher ist, einen Spurwechsel durchzuführen. Plötzlich nähert sich ein Radfahrer, welcher vorher nicht erfasst wurde. Durch den sich ständig wiederholenden \textit{sense-plan-act}-Prozess kann das Fahrzeug seinen ursprünglichen Plan entsprechend anpassen und den Spurwechsel abbrechen. Der Fakt, dass der Prozess des Erkennens und Anpassens, sich in einem ständigen Kreislauf befindet, führt dazu, dass sich autonome Fahrzeuge sicher im Straßenverkehr bewegen können.
\vgl[142]{Surden}
