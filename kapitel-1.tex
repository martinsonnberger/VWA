\chapter{Autonomes Fahren --- Beschreibung allgemein}

Versucht man den Begriff \enq{Autonomes Fahren} zu definieren, stößt man unweigerlich auf das Automobil und dessen Erfindung im Jahr 1886. Der Begriff \enq{Automobil} setzt sich aus dem griechischen \textit{autòs} (\enq{selbst}) und dem lateinischen \textit{mobilis} (\enq{beweglich}) zusammen, bedeutet also \enq{Selbstbewegliche}. Damals sollte dadurch, die von nun an bestehende Unabhängigkeit von Kutschpferden verdeutlicht werden. Genauer betrachtet fällt auf, dass durch das Wegfallen von Pferden die Autonomie in gewisser Weise verloren ging. Durch jahrelanges Training konnten sich die Pferde in Maßen selbst fortbewegen, indem sie beispielsweise einen nicht mehr fahrtüchtigen Kutscher alleine nachhause brachten.
\vgl[2-3]{Maurer}
Durch die Einführung von zahlreichen Fahrerassistenzsystemen und letzendlich des autonomen Fahrens wird dem Automobil seine ursprünglich gedachte Autonomie nicht nur wieder zurückgegeben, sondern auch um ein Vielfaches erweitert.

\section{Begriffsdefintion}

\dzit{Definition}{Autonomes Fahren bedeutet das selbständige, zielgerichtete Fahren eines Fahrzeugs im realen Verkehr, ohne Eingriff des Fahrers.}

Autonomes Fahren beschreibt die Übergabe aller mit der Fahrzeugführung verbundenen Tätigkeiten wie beschleunigen, bremsen, lenken, etc., an eine Maschine, den sogenannten Fahrroboter, welcher selbstständig, ohne Mitwirken des Menschen, Entscheidungen trifft. Dabei darf von autonomen Fahrzeugen keine größere Gefahr, als von Menschen gesteuerten ausgehen. Primäre Informationsquellen sind visuelle Quellen, wie sie auch dem Fahrer zur Verfügung stehen.
\vgls[667]{Winner}{Definition}

Ein Punkt, in dem sich autonom gesteurte Fahrzeuge von heutigen Maschinen unterscheiden, ist die Unvorhersehbarkeit ersterer. Maschinen im heutigen Alltag werden entweder von Menschen gesteuert (\zB Autos, Baumaschinen), oder führen stark eingeschränkte, sich wiederholende Bewegungen durch (\zB Aufzüge, Rolltreppen). Autonome Fahrzeuge hingegen bewegen sich auf normalen Straßen und fahren auf beinahe allen Wegen, die auch ein menschlicher Fahrer benutzen würde. Zusammen mit der alleinig durch den Computer stattfindenden Entscheidungsfindung, lässt sie das unvorhersehbar machen.

Selbstfahrende Autos stellen eine neue Klasse an von Computern kontrollierten Systemen dar, in der die Maschine Entscheidungen für sich selbst trifft. Um solche Systeme zu verstehen, ist es umso wichtiger, ein Verständnis der dahintersteckenden Technologie von autonomen Fahrzeugen zu haben, die es ihnen erlaubt, sich selbst durch das komplexe Straßennetz zu manövrieren.
\vgl[130]{Surden}

\subsection{Autonomiestufen}
Der Grad der Automatisierung eines Fahrzeugs kann in verschiedene Stufen eingeteilt werden, welche Autonomiestufen oder Level genannt werden. Die Einteilung erfolgt in sechs Stufen, welche in Europa \ua von der deutschen Bundesanstalt für Straßenwesen und in den USA von der SAE International festgelegt werden. (Siehe \ref{levels})

Am Beginn der Skala steht Level 0, was bedeutet, dass sämtliche Fahraufgaben durch den Menschen erfolgen. Level 5 hingegen steht für eine vollständige Automation, bei der jegliche, für eine Fahrt erforderlichen Aufgaben, durch den Computer ausgeführt werden. Bei Begriffen wie \enq{selbstfahrend}, \enq{fahrerlos} oder auch \enq{autonom}, ist meist von einer vollständigen Automation auf Level 5 die Rede.

%TODO: Fragen ob das stimmt
\nocite{wiki-levels}
\begin{table}[p]
    \begin{tabularx}{\textwidth}{P{1.2cm}lX}
      \toprule

      \textbf{SAE Level} & \textbf{Name} & \textbf{Definition} \\

      \midrule
      \multicolumn{3}{l}{Menschlicher Fahrer überwacht die Umgebung.} \\
      \midrule

      0 & Keine Automation & die durchgängige Ausführung aller Aspekte der dynamischen Fahraufgabe durch den menschlichen Fahrer, auch wenn unterstützende Warn- oder Interventionssysteme eingesetzt werden. \\[0.3cm]

      1 & Fahrerassistenz & eine Fahrer-Assistenz, die Fahrmodus-spezifische Aufgaben wie die Lenk-Assistenz oder Beschleunigungs- und Brems-Assistenz dank der Verwendung von Fahr- und Umgebungsinformationen ausführt und mit der Erwartung, dass der menschliche Fahrer alle verbleibenden Aspekte der dynamischen Fahraufgabe ausführt \\[0.3cm]

      2 & Teilautomation & die Fahrmodus-spezifische Ausführung von Lenk- und Beschleunigungs- und Bremsvorgängen durch ein oder mehrere Fahrerassistenzsysteme unter Verwendung von Informationen über die Fahrumgebung und mit der Erwartung, dass der menschliche Fahrer alle verbleibenden Aspekte der dynamischen Fahraufgabe ausführt \\

      \midrule
      \multicolumn{3}{l}{Das System überwacht die Umgebung.} \\
      \midrule

      3 & Bedingte Automation & die Fahrmodus-spezifische Ausführung aller Aspekte der dynamischen Fahraufgabe durch ein automatisiertes Fahrsystem mit der Erwartung, dass der menschliche Fahrer auf Anfrage des Systems angemessen reagieren wird \\[0.3cm]

      4 & Hohe Automation & die Fahrmodus-spezifische Ausführung aller Aspekte der dynamischen Fahraufgabe durch ein automatisiertes Fahrsystem, selbst wenn der menschliche Fahrer auf Anfrage des Systems nicht angemessen reagiert \\[0.3cm]

      5 & Volle Automation & die durchgängige Ausführung aller Aspekte der dynamischen Fahraufgabe durch ein automatisiertes Fahrsystem unter allen Fahr- und Umweltbedingungen, die von einem menschlichen Fahrer bewältigt werden können \\

      \bottomrule
    \end{tabularx}
  \caption[Einteilung der Autonomiestufen nach SAE J3016. Quelle: \fullcite{sae-j3016}.]{Einteilung der Autonomiestufen nach SAE J3016 (\quotecite(vgl.)()[19]{sae-j3016}[]{wiki-levels})}
  \label{levels}
\end{table}



\section{Funktionsweise}

%TODO: Sekundärzitat: Quelle finden!
Bei einem hohen Automatisierungsgrad wird die Technologie verwendet, um die folgenden drei primären Fragen zu beantworten:
\begin{enumerate}
  \item{Wo befindet sich das Fahrzeug?}
  \item{Welche Objekte befinden sich um das Fahrzeug herum?}
  \item{Wo ist es gewünscht, erlaubt und sicher, sich als nächstes hinzubewegen?}
\end{enumerate}
\vgl[137]{Surden}

Einem autonomen Fahrzeug muss es möglich sein, diese Fragen unter Berücksichtung von Verkehrszeichen, Ampeln und Straßenmarkierungen, zu beantworten. Außerdem muss es in einem Raum voll mit Objekten wie Fahrzeugen, Fußgängern, Radfahrern und Tieren, sicher navigieren können. Um diese Aufgaben erfolgreich zu bewältigen, ist ein Zusammenspiel von \textbf{Hardware} und \textbf{Software} notwendig.

\subsection{Hardware}

\subsection{Software}
