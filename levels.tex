\begin{table}
    \begin{tabularx}{\textwidth}{P{1.2cm}lX}
      \toprule

      \textbf{SAE Level} & \textbf{Name} & \textbf{Definition} \\

      \midrule
      \multicolumn{3}{l}{Menschlicher Fahrer kontrolliert die Umgebung.} \\
      \midrule

      0 & Keine Automation & die durchgängige Ausführung aller Aspekte der dynamischen Fahraufgabe durch den menschlichen Fahrer, auch wenn unterstützende Warn- oder Interventionssysteme eingesetzt werden. \\[0.3cm]

      1 & Fahrerassistenz & eine Fahrer-Assistenz, die Fahrmodus-spezifische Aufgaben wie die Lenk-Assistenz oder Beschleunigungs- /Brems-Assistenz dank der Verwendung von Fahr- und Umgebungsinformationen ausführt und mit der Erwartung, dass der menschliche Fahrer alle verbleibenden Aspekte der dynamischen Fahraufgabe ausführt \\[0.3cm]

      2 & Teilautomation & die Fahrmodus-spezifische Ausführung von Lenk- und Beschleunigungs- /Bremsvorgängen durch ein oder mehrere Fahrerassistenzsysteme unter Verwendung von Informationen über die Fahrumgebung und mit der Erwartung, dass der menschliche Fahrer alle verbleibenden Aspekte der dynamischen Fahraufgabe ausführt \\

      \midrule
      \multicolumn{3}{l}{Das System kontrolliert die Umgebung.} \\
      \midrule

      3 & Bedingte Automation & die Fahrmodus-spezifische Ausführung aller Aspekte der dynamischen Fahraufgabe durch ein automatisiertes Fahrsystem mit der Erwartung, dass der menschliche Fahrer auf Anfrage des Systems angemessen reagieren wird \\[0.3cm]

      4 & Hohe Automation & die Fahrmodus-spezifische Ausführung aller Aspekte der dynamischen Fahraufgabe durch ein automatisiertes Fahrsystem, selbst wenn der menschliche Fahrer auf Anfrage des Systems nicht angemessen reagiert \\[0.3cm]

      5 & Volle Automation & die durchgängige Ausführung aller Aspekte der dynamischen Fahraufgabe durch ein automatisiertes Fahrsystem unter allen Fahr- und Umweltbedingungen, die von einem menschlichen Fahrer bewältigt werden können \\

      \bottomrule
    \end{tabularx}
  \caption[Einteilung der Autonomiestufen nach SAE J3016. Quelle: \fullcite{wiki-levels}.]{Einteilung der Autonomiestufen nach SAE J3016}
  \label{levels}
\end{table}
