\addchap{Abstract}

In dieser vorwissenschaftlichen Arbeit wird der Prozess des autonomen Fahrens genauer erläutert. Insbesondere wird die Frage beantwortet, welche Methoden angewandt werden, um die Position eines autonomen Fahrzeugs präzise bestimmen zu können.
\bigskip

Im ersten Teil der Arbeit wird der Begriff des autonomen Fahrens und die verschiedenen Stufen, in die die Autonomie eines Fahrzeugs eingeteilt werden kann, erklärt. Außerdem werden die im Fahrzeug verbauten Sensoren sowie der Prozess des autonomen Fahrens behandelt. Fortgesetzt wird die Arbeit mit den verschiedenen Methoden der Lokalisierung, abschließend wird der aktuelle Entwicklungsstand, sowohl von Produkten zur Positionsbestimmung als auch von Projekten zum autonomen Fahren, dargelegt.
\bigskip

Nach intensiver Auseinandersetzung mit dem Thema habe ich die Erkenntnis gewonnen, dass noch keine der in dieser Arbeit behandelten Technologien eine endgültige Lösung für das Problem der notwendigen genauen Positionsbestimmung darstellt. Letztendlich müssen die Fahrzeughersteller wohl auf eine Kombination verschiedener Systeme zurückgreifen, um den hohen Anforderungen gerecht zu werden.
