% Vorlage für die VWA. Version 20180117 (c) Leonard Michlmayr

% TODO: Wähle den für deine Arbeit die passenden Optionen!
\documentclass[DLS,
	inreferencehack,
	ohneVgl=false,
	ohneS=false,
	noscauthor,
	rundeauslassung=false,
	bookstyle=false,
	widowlines=3]{vwa}

%% Chapter Style
\usepackage{graphicx}
\usepackage{titlesec}

\newcommand*\HUGE{\Huge}
\newcommand*\chapnamefont{\normalfont\LARGE\MakeUppercase}
\newcommand*\chapnumfont{\normalfont\HUGE}
\newcommand*\chaptitlefont{\normalfont\HUGE\bfseries}

\newlength\beforechapskip
\newlength\midchapskip
\setlength\midchapskip{\paperwidth}
\addtolength\midchapskip{-\textwidth}
\addtolength\midchapskip{-\oddsidemargin}
\addtolength\midchapskip{-1in}
\setlength\beforechapskip{18mm}

\titleformat{\chapter}[display]
  {\normalfont\filleft}
  {{\chapnamefont\chaptertitlename}%
    \makebox[0pt][l]{\hspace{.8em}%
      \resizebox{!}{\beforechapskip}{\chapnumfont\thechapter}%
      \hspace{.8em}%
      \rule{\midchapskip}{\beforechapskip}%
    }%
  }%
  {25pt}
  {\chaptitlefont}
\titlespacing*{\chapter}
  {0pt}{40pt}{40pt}

%% Abkürzungen
\newcommand{\zB}{z.\,B. }
\newcommand{\ua}{u.\,a. }
\newcommand{\dH}{d.\,h. }

%% Trick texlipse to use biber instead of bibtex
\iffalse
\usepackage[error,backend=biber]{biblatex}
\fi
%%

\usepackage{textcomp}
\usepackage[output-decimal-marker={,}]{siunitx}
\usepackage{tabularx}
\usepackage{booktabs}
\usepackage{acronym}
\usepackage{mhchem}

\usepackage{amsmath}
\DeclareMathOperator{\arctantwo}{arctan2}

\usepackage{array}
\newcolumntype{P}[1]{>{\centering\arraybackslash}p{#1}}

\usepackage{hyperref}

\renewcommand{\autodot}{}

% TODO: Link visit date aktualisieren
\addbibresource{quellen.bib}
\graphicspath{{img/}}

\Autor{Martin Sonnberger}
\Klasse{8C}
\Betreuer{Mag. Leonard Michlmayr}
\Thema{Positionsbestimmung beim autonomen Fahren}

% TODO: erst bei der letzten Version das Abgabedatum anführen
%\Abgabedatum{}

\begin{document}

% Am Anfang keine Seitennummern
\frontmatter

% PDF-Lesezeichen für die Titelseite
\pdfbookmark[0]{Titelseite}{titlepage}
% Titelseite
\maketitle

\addchap{Abstract}

In dieser vorwissenschaftlichen Arbeit wird der Prozess des autonomen Fahrens genauer erläutert. Insbesondere wird die Frage beantwortet, welche Methoden angewandt werden, um die Position eines autonomen Fahrzeugs präzise bestimmen zu können.
\bigskip

Im ersten Teil der Arbeit wird der Begriff des autonomen Fahrens und die verschiedenen Stufen, in die die Autonomie eines Fahrzeugs eingeteilt werden können, erklärt. Außerdem werden die im Fahrzeug verbauten Sensoren sowie der Prozess des autonomen Fahrens behandelt. Fortgesetzt wird die Arbeit mit den verschiedenen Methoden der Lokalisierung, abschließend wird der aktuelle Entwicklungsstand, sowohl von Produkten zur Positionsbestimmung als auch von Projekten zum autonomen Fahren, dargelegt.
\bigskip

Nach intensiver Auseinandersetzung mit dem Thema habe ich die Erkenntnis gewonnen, dass noch keine der in dieser Arbeit behandelten Technologien eine endgültige Lösung für das Problem der notwendigen genauen Positionsbestimmung darstellt. Letztendlich müssen die Fahrzeughersteller wohl auf eine Kombination verschiedener Systeme zurückgreifen um den hohen Anforderungen gerecht zu werden.


% TODO: eventuell Vorwort einfügen
% \include{vorwort}

% Das Inhaltsverzeichnis soll ein PDF-Lesezeichen aber keinen Eintrag im
% Inhaltsverzeichnis haben.
\cleardoublepage\pdfbookmark[0]{\contentsname}{toc}
% Inhaltsverzeichnis
\tableofcontents

% Hier geht es los.
\mainmatter

\chapter{Einleitung}

Autonomes Fahren, ein Thema, das unsere Zukunft massiv verändern wird. Benötigt man noch ein eigenes Auto, um zur Arbeit zu fahren? Oder bestellt man einfach ein autonomes Taxi über eine App? Oder noch besser, kommt das autonome Taxi täglich um 7 Uhr vor die Haustür, weil es weiß, dass man um diese Uhrzeit zur Arbeit fahren muss? Das alles sind Fragen, die uns in einigen Jahren sicherlich beschäftigen werden, sich heute noch nicht genau beantworten lassen können.

Es ist wichtig, vor der neuen Technologie nicht zurückzuschrecken, sondern ihr eine Chance zu geben, unser Leben einfacher und vor allem sicherer zu gestalten. Elon Musk sagte: \zit{elon-musk-quote}{When Henry Ford made cheap, reliable cars people said, \enq{Nah, what's wrong with a horse?} That was a huge bet he made, and it worked.}

\bigskip

Diese Vorwissenschaftliche Arbeit soll die Technologie, die hinter autonomen Fahrzeugen steckt, erläutern, um die Funktionsweise der Autos, denen wir künftig unser Leben anvertrauen werden, ein wenig besser zu verstehen und zu durchblicken.

Das erste Kapitel dieser Literaturarbeit liefert einen Überblick über die allgemeine Funktionsweise autonomer Fahrzeuge. Außerdem werden die verschiedenen Autonomiestufen erklärt, die Fahrzeuge in ihren Grad der Automation einteilen.

Im zweiten Kapitel wird besonders auf die Positionsbestimmung autonomer Fahrzeuge eingegangen, indem die verschiedenen Methoden der Lokalisierung genauer erläutert werden.

Das dritte Kapitel behandelt den aktuellen Stand der Dinge und erklärt, was heute bereits mit autonomen Fahrzeugen möglich ist. Außerdem wird versucht, einen Blick in die Zukunft zu werfen und eine Schätzung abzugeben wann wir in einer Welt voller autonomer Fahrzeuge leben werden.


\chapter{Autonomes Fahren --- Beschreibung allgemein}
\section{Begriffsdefintion}
Um den Begriff \enq{Autonomes Fahren} definieren zu können, muss in der Zeit ein Sprung zurück gemacht werden, nämlich zur Erfindung des Automobils im Jahr 1886. Der Begriff \enq{Automobil} setzt sich aus dem griechischen \textit{autòs} (\enq{selbst}) und dem lateinischen \textit{mobilis} (\enq{beweglich}) zusammen, bedeutet also \enq{Selbstbewegliche}. Damals sollte dadurch, die von nun an bestehende Unabhängigkeit von Pferden verdeutlicht werden. Genauer betrachtet fällt auf, dass durch das Wegfallen von Kutschpferden die Autonomie in gewisser Weise verloren ging. Durch jahrelanges Training konnten sich die Pferde in Maßen selbst fortbewegen, indem sie beispielsweise einen nicht mehr fahrtüchtigen Kutscher alleine nachhause brachten.

Durch die Einführung von zahlreichen Fahrerassistenzsystemen und letzendlich des autonomen Fahrens wird dem Automobil seine ursprünglich gedachte Autonomität nicht nur wieder zurückgegeben, sondern auch um ein Vielfaches erweitert.

\zit{Definition}{Autonomes Fahren bedeutet das selbständige, zielgerichtete Fahren eines Fahrzeugs im realen Verkehr, ohne Eingriff des Fahrers.}

\section{Automatisierungsgrade}
Der Grad der Automatisierung kann in verschiedene Stufen eingeteilt werden.

\chapter{Positionsbestimmung}\label{kapitel-2}

Dieses Kapitel behandelt die Frage, wie es autonome Fahrzeuge schaffen, sich selbst in unterschiedlichsten Verkehrssituationen auf wenige \si{\centi\meter} genau zu lokalisieren. Gerade im Verkehr ist eine besonders hohe Genauigkeit gefragt, eine ungefähre Standortbestimmung ist für autonome Fahrzeuge definitiv nicht ausreichend.


\section{Satellitenortung}

Das wohl bekannteste heute eingesetzte Ortungssystem, welches auf Satellitenortung basiert, ist das \ac{GPS}, welches vom US-Verteidigungsministerium betrieben wird. \vgl{Hilty} In etwa \SI{20000}{\kilo\meter} Höhe umkreisen 31 Satelliten zweimal täglich die Erde und senden Funksignale auf aus, welche Informationen zu Position und Uhrzeit des Satelliten enthalten. Die Genauigkeit bei einer \ac{GPS}-Ortung liegt bei ungefähr 5 \si{\meter}.
\vgl{Winner}

Das russische GLONASS und das europäische Galileo-Projekt sind zwei weitere satellitenbasierte Ortungssysteme. Ein wichtiger Vorteil dieser beiden Systeme liegt darin, dass in Kombination mit \ac{GPS} und damit verbunden einer größeren Satellitenzahl, eine höhere Genauigkeit und Zuverlässigkeit der Ortung erreichbar ist.
\vgl{Hilty}

\subsection{Berechnung des Standortes}

Die Positionsbestimmung erfolgt bei \ac{GPS} durch Entfernungsmessung zu mehreren Satelliten. Ein Satellit ist nicht ausreichend, da sich der Empfänger überall auf der Oberfläche einer Kugel befinden kann, deren Mittelpunkt der Standort des Satelliten und Radius die Entfernung zwischen Sender und Empfänger ist. Um alle drei Koordinaten (x, y, z) berechnen zu können, sind Entfernungen zu drei Satelliten notwending, um die erforderlichen Gleichungen aufstellen zu können.

Die Entfernungsmessung geschieht durch Messung der Laufzeit des Signals (\dH die Zeit, die das Signal für die Wegstrecke zwischen Sender und Empfänger benötigt). Das Signal breitet sich mit Lichtgeschwindigkeit aus, dadurch lässt sich die Distanz zum Satelliten berechnen.

Allerdings ist dem Empfänger zunächst nicht bekannt, wann das Signal den Sender verlassen hat. Da Sender und Empfänger bei \ac{GPS} nicht direkt miteinander kommunizieren können, spricht man von einer Einweg-Entfernungsmessung. Der Satellit schickt die Sendezeit des Siganls als Code mit, nämlich als \ac{GPS}-Systemzeit im Moment der Sendung. Da der Empfänger allerdings nicht mit der \ac{GPS}-Systemzeit synchronisiert ist, entsteht eine zusätzliche vierte Unbekannte, wodurch eine vierte Gleichung und somit ein vierter Satellit notwendig ist.

Zur Vereinfachung wird die Ausbreitungsgeschwindigkeit als konstant angenommen. Die Gleichungen ergeben sich durch Gleichsetzung der vier Entfernungen in Koordinaten und den Distanzen aus der Laufzeitmessung. Um keine Wurzeln zu benötigen, schreibt man die Gleichungen in Quadratform.
\vgl{wiki-gps}
\begin{align}
  (x_1 - x_0)^2 + (y_1 - y_0)^2 + (z_1 - z_0)^2 = [c(t_1 - t_0)]^2 \\
  (x_2 - x_0)^2 + (y_2 - y_0)^2 + (z_2 - z_0)^2 = [c(t_2 - t_0)]^2 \\
  (x_3 - x_0)^2 + (y_3 - y_0)^2 + (z_3 - z_0)^2 = [c(t_3 - t_0)]^2 \\
  (x_4 - x_0)^2 + (y_4 - y_0)^2 + (z_4 - z_0)^2 = [c(t_4 - t_0)]^2
\end{align}

Löst man dieses Gleichungssystem erhält man die Werte für den Sendezeitpunkt \(t_0\) und die Koordinaten des Empfängers \(x_0, y_0, z_0\). Diese kartesischen Koordinaten lassen sich mithilfe folgender Formeln in Koordinaten mit Längen- und Breitengrad umwandeln:
\begin{align}
  Breite &= \arcsin(\frac{z}{R}) \\
  Länge &= \arctantwo(y, x)
\end{align}

R = Erdradius


\subsection{Verwendung in autonomen Fahrzeugen}

Für autonome Fahrzeuge ist die \ac{GPS}-Ortung nur von relativ geringer Bedeutung, da deren Genauigkeit nicht für autonomes Fahren ausreichend ist. \ac{GPS} wird bei autonomen Fahrzeugen lediglich für eine erste grobe Schätzung des Standorts verwendet. Für eine genauere Ortung kommen \acs{Lidar}-Sensoren sowie \enq{Map-Matching} zum Einsatz.


\section{Lidar-Technik}

\acs{Lidar}-Sensoren (\acl{Lidar}) verwenden Laserstrahlen in einem für Menschen nicht sichtbaren Frequenzbereich, um die Umgebung von autonomen Fahrzeugen ständig zu überwachen. Dazu werden bis zu \num{150000} Impulse pro Sekunde ausgesendet, welche von Objekten reflektieren und anschließend vom \acs{Lidar}-Modul wieder empfangen werden. \vgl{lidar-radar} Mithilfe der gemessenen Laufzeit, die das Licht für die Strecke zum Objekt und wieder zurück benötigt hat, lässt sich die Distanz zum Objekt berechnen.

\begin{equation}
  Distanz = \frac{Laufzeit \cdot Lichtgeschwindigkeit}{2}
\end{equation}
\vspace{\baselineskip}

Zur Positionsbestimmung werden \acs{Lidar}-Sensoren einem \acs{Radar} vorgezogen, da sie eine höhere Auflösung und damit detailliertere Abbildungen der Umgebung erstellt werden können, wie in \ref{lidar-radar} zu sehen ist.

\begin{figure}\centering
  \includegraphics[width=\textwidth]{lidar-radar.png}
  \captionbelow[Vergleich Lidar und Radar Umgebungsscan. Quelle: \fullcite{lidar-radar})]{Vergleich Lidar und Radar Umgebungsscan (\cite{lidar-radar})}
  \label{lidar-radar}
\end{figure}

In den Fahrzeugen von Waymo, einem Tochterkonzern von Google's Mutterkonzern Alphabet, sind drei \acs{Lidar}-Scanner verbaut. Dazu gehören ein Kurzstrecken-\acs{Lidar} für den unmittelbaren Bereich um das Fahrzeug, ein Langstrecken-\acs{Lidar}, der auch aus mehreren hundert Metern Entfernung und voller Geschwindigkeit feine Signale, wie Handbewegungen, erkennen kann sowie einem hoch auflösenden \acs{Lidar}, welcher Millionen Laserimpulse aussenden kann, um ein detailliertes Bild der Umwelt zu erzeugen. \vgl{waymo} In autonomen Fahrzeugen wird \acs{Lidar} hauptsächlich für zwei Aufgaben eingesetzt:
\begin{enumerate}
  \item{Erkennung und Bestimmung von Objekten um das Fahrzeug}
  \item{Präzise Positionsbestimmung innerhalb der Fahrspur}
\end{enumerate}
\vgl{Surden}

Um das Fahrzeug präzise lokalisieren zu können, scannen \acs{Lidar}-Sensoren Straßenmarkierungen, indem sie die Straßenoberfläche auf unterschiedliche Reflexionsvermögen (aufgrund weißer Markierungen) untersuchen und speichern die Daten in einer Datei namens \acs{BLMR} (\acl{BLMR}). In der \acs{BLMR} werden stets die Leit- und Randliniendaten der letzten \SI{240}{\meter} gespeichert, diese werden laufend mit den zuvor erstellten annotierten digitalen Karten (siehe \ref{maps}) abgeglichen, wodurch der Standort bestimmt werden kann. \vgl{self-localization}

Sind keine Straßenmarkierungen vorhanden, können sich die Sensoren automatisch an anderen markanten Punkten wie Straßenschildern, Ampelanlagen, Gebäuden oder Bäumen, orientieren. Wichtig ist nur, dass es sich bei den Objekten um Dinge handelt, deren Standort sich nicht ändert und die eindeutig für \acs{Lidar}-Sensoren erkennbar sind.

\subsection*{Nachteile von Lidar}

Trotz der Vorteile wie der hohen Genauigkeit oder auch großen Reichweite haben auch \acs{Lidar}-Scanner Nachteile. Einer davon ist der Preis, besonders Geräte für große Distanzen sind teuer, da hierfür Laser mit \SI{1550}{\nano\meter} Wellenlänge benötigt werden, um für Menschen nicht schädlich zu sein. Um solche Laser wiederum empfangen zu können, sind Empfänger aus \ac{InGaAs} notwendig, da Siliziumempfänger, welche um ein Vielfaches günstiger als \ac{InGaAs} sind, keine Laserstrahlen mit \SI{1550}{\nano\meter} erkennen können. \vgl{wired} Durch jahrelange Entwicklung und ständig steigenden Produktionszahlen hat es Waymo mittlerweile geschafft, die Kosten für einen \acs{Lidar}-Scanner von \$\,\num{75000} um 90\,\% auf \$\,\num{7500} zu reduzieren. \vgl{waymo-medium}

Ein weiterer Nachteil ist die Abhängigkeit von gutem Wetter. Schnee, Regen oder Nebel reflektieren das Licht, was zur Folge hat, dass das Fahrzeug fälschlicherweise Hindernisse erkennt, welche in Wirklichkeit nur Regentropfen oder Schneeflocken sind. Dieser Bereich unterliegt aktuell noch weiteren Forschungen, Ford hat jedoch schon einen Algorithmus entwickelt, der dieses Problem beheben soll. Dabei werden die empfangenen Laserstrahlen genau auf ihre Eigenschaft untersucht, etwa ob sich ein Objekt zweimal an der selben Stelle befindet, was bei Regentropfen nicht der Fall ist. \vgl{ford-qz}

\chapter{Aktueller Entwicklungsstand}

Dieses Kapitel soll einen Überblick über derzeitige Projekte schaffen, die es sich zur Aufgabe gemacht haben, autonomes Fahren weltweit auf die Straßen zu bringen.

\section{Waymo}

Waymo ist ein Schwesterunternehmen von Google und führt das im Jahr 2009 gegründete \textit{Google Driverless Car}-Projekt fort. 2015 fand die erste, voll autonome Fahrt auf öffentlichen Straßen in Austin statt. Im Oktober 2018 erreichte Waymo den Meilenstein von 10 Millionen zurückgelegten Meilen (ca. 16 Millionen Kilometer) in ihrer Flotte von autonomen Fahrzeugen.

Am 5.\ Dezember 2018 startete Waymo ein kommerzielles Taxi-Service namens \textit{Waymo One} in vier Vororten von Phoenix, Arizona. Die autonomen Fahrzeuge werden derzeit noch von einem \enq{Sicherheitsfahrer} begleitet, der im Fall einer Fehlfunktion der autonomen Systeme, die Kontrolle über das Fahrzeug übernehmen kann.
\vgls{waymo-taxiservice} Bereits seit 2017 bis zum Start von \textit{Waymo One} lief das sogenannte \textit{Early rider program}, an dem ausgewählte Bewohner des Ballungsraums von Phoenix teilnehmen konnten, um kostenlose Taxifahrten zu erhalten. Das Feedback der Teilnehmer trug zur Weiterentwicklung der autonomen Fahrzeuge bei.
\vgl{waymo-website}


\section{Tesla}

Tesla ist, ebenso wie Waymo, ein im Silicon Valley ansässiges Unternehmen mit folgendem selbst ernannten Ziel: \zit{tesla-about}{Die Beschleunigung des Übergangs zu nachhaltiger Energie.} Um dieses Ziel erreichen zu können, stellte Tesla im Jahr 2008 den \enq{Roadster} vor, einen zweisitzigen Sportwagen, der die Finanzierung der Premium-Limousine \enq{Model S} sicherte. Besondere Merkmale sind die größte Reichweite unter Elektrofahrzeugen und eine Beschleunigung von 0 auf 100 \si[per-mode=symbol]{\kilo\metre\per\hour} in 2,7 Sekunden. Derzeit läuft die Auslieferung der im Vergeleich zum \enq{Model S} kompakteren Limousine \enq{Model 3}, welches durch einen Grundpreis von \num{35000} US-Dollar die Verbreitung von Elektrofahrzeugen weiter vorantreiben soll.
\vgl{tesla-about}

Teslas Fahrzeuge zählen außerdem zu den sichersten der Welt. Wie in \ref{safety-chart} zu sehen, sind von allen seit 2011 von der \ac{NHTSA} getesteten Fahrzeuge, die drei mit der niedrigsten Wahrscheinlichkeit von Verletzungen, alle von Tesla hergestellt. Das \enq{Model 3} hat hierbei einen besonders niedrigen Wert von nur 5,7 Prozent. Dieser Wert errechnet sich durch den \enq{Vehicle Safety Score} von 0,38 der mit der von der \ac{NHTSA} angegebenen Basiswahrscheinlichkeit von 15 Prozent multipliziert wird. \vgls[40038]{nhtsa-baseline}{model-3-score}

\begin{figure}\centering
  \includegraphics[width=\textwidth]{safety-chart.jpg}
  \captionbelow[Niedrigste Wahrscheinlichkeit von Verletzungen. Getestet von der \ac{NHTSA}. Bildquelle: \fullcite{tesla-safety}]{Niedrigste Wahrscheinlichkeit von Verletzungen. Getestet von der \ac{NHTSA} (\cite{tesla-safety})}
  \label{safety-chart}
\end{figure}

Weiters ist Tesla für seinen bereits in \ref{section-2-3} behandelten \textit{Autopilot}. Am 27.\ November haben Teslas Fahrzeuge eine Milliarde Meilen (ca. 1,6 Milliarden Kilometer) mit aktiviertem \textit{Autopilot} zurückgelegt. Das entspricht etwa 10 Prozent aller von Teslas Fahrzeuges weltweit gefahrenen Meilen, wobei hierbei auch Fahrzeuge ohne \textit{Autopilot}-Hardware bzw. solche mit Hardware, jedoch ohne der notwendigen kostenpflichtigen Aktivierung des Assistenzsystems miteingerechnet werden. \vgl{electrek-one-billion}


\section{Wiener Linien}

Im Frühjahr 2019 soll die erste autonome Buslinie Wiens in der Seestadt in Betrieb gehen. Das Projekt namens \textit{auto.Bus - Seestadt} durchläuft derzeit intensive Testfahrten auf geschlossenem Gelände. Bei dem Bus handelt es sich um einen vollelektrischen Kleinbus (siehe \ref{bus-seestadt}), der Platz für bis zu zehn Fahrgäste und einen Operator bietet. Der Operator ist derzeit noch aus sicherheitsgründen notwendig, er überwacht den Bus und kann im Ernstfall ins Geschehen eingreifen. \vgl{wiener-linien}

\begin{figure}\centering
  \includegraphics[width=\textwidth]{bus-seestadt.jpg}
  \captionbelow[Elektrischer Kleinbus der Wiener Linien. Bildquelle: \fullcite{wiener-linien}]{Elektrischer Kleinbus der Wiener Linien (\cite{wiener-linien})}
  \label{bus-seestadt}

  \includegraphics[width=\textwidth]{bus-seestadt-karte.jpg}
  \captionbelow[Geplante Route der autonomen Buslinie. Bildquelle: \fullcite{wiener-linien}]{Geplante Route der autonomen Buslinie (\cite{wiener-linien})}
  \label{bus-seestadt-karte}
\end{figure}

Die geplante Route ist in \ref{bus-seestadt-karte} dargestellt, sie ist 2 \si{\kilo\metre} lang und bedient sechs Stationen in der Seestadt. Die Beförderung auf der Linie ist kostenlos. Da es sich um einen Testbetrieb handelt, ist der Transport von Kinderwägen und Rollstühlen rechtlich nicht erlaubt.
\vgl{wiener-linien}


\section{Aktionspaket des Bundesministeriums für Verkehr, Innovation und Technologie}

Das \ac{bmvit} veröffentliche im November 2018 sein \textit{Aktionspaket Automatisierte Mobilität. 2019 -- 2022}, in dem Österreichs Ziele und Maßnahmen zur Verbreitung von autonomem Fahren dargelegt werden.

Durch eine Novelle des Kraftfahrgesetztes im Jahr 2016, wurden die rechtlichen Rahmenbedingungen geschaffen, um \enq{ALP.Lab} (Austrian Light Vehicle Proving Region for Automated Driving), Österreichs erste Testumgebung für autonome Fahrzeuge, in der Steiermark zu errichten. Durch Erstellung der Verordnung zum automatisierten Fahren\footnote{Automatisiertes Fahren Verordnung - AutomatFahrV. Online abrufbar unter: \url{https://www.ris.bka.gv.at/GeltendeFassung.wxe?Abfrage=Bundesnormen&Gesetzesnummer=20009740}} konnten außerdem erstmals Test auf öffentlichen Straßen durchgeführt werden. Hierbei werden insbesondere Autobahnpiloten mit automatischem Spurwechsel, selbstfahrende Heeresfahrzeuge und autonome Kleinbusse getestet.
\vgl[10]{bmvit}

Das \ac{bmvit} strebt einen \zit[23]{bmvit}{verkehlich sinnvollen und effizienten Einsatz automatisierter Mobilität sowie die Stärkung der Wettbewerbsposition Österreichs} an. In erster Linie gehe es allerdings laut Aktionsplan um \zit[23]{bmvit}{lebenswerte öffentliche Räume.} Diese Punkte will die Regierung besonders durch Testen und Pilotieren erreichen, wobei der Nutzer Mittelpunkt dieser Tests sein soll. Letztendlich muss nämlich der Verkehr für alle die an ihm teilnehmen eines werden: sicherer. Insbesondere schwächere Verkehrsteilnehmer wie Radfahrer oder Fußgänger sollen in ihrer Sicherheit gestärkt werden.

Ein weiteres wichtiges Anliegen ist die Reduktion der \ce{CO2}-Emissionen im Verkehrssektor, bis 2050 soll Österreichs Verkehr vollständig \ce{CO2}-neutral werden. Das ist nicht nur durch den Einsatz elektrischer Fahrzeuge erreichbar, sondern auch durch autonome Mobilität. Die Notwendigkeit eines eigenen Fahrzeuges ist durch autonome Taxis, die rund um die Uhr von jedermann verwendet werden können, nicht mehr so hoch wie heute.
\vgl[24]{bmvit}


\chapter{Zusammenfassung und Fazit}

Diese Arbeit hat die Frage behandelt, welche Methoden zur Positionsbestimmung bei autonomen Fahrzeugen angewandt werden.

Durch Auseinandersetzung mit dem Thema bin ich zur Erkenntnis gekommen, dass die Hersteller autonomer Fahrzeuge derzeit noch auf viele verschiedene Systeme zur Lokalisierung der Fahrzeuge setzen. Noch ist keine der Technologien ausgereift genug, um massentaugliche und erschwingliche autonome Fahrzeuge für den Alltag herstellen zu können. \acs{Lidar} arbeitet auf einem hohen Genauigkeitslevel, ist jedoch noch zu teuer und auch nicht platzsparend. \ac{GPS} hingegen ist kostengünstig, allerdings für eine präzise Positionsbestimmung nicht ausreichend genau.

Letztendlich ist es eine Kombination der Systeme, die ein gutes Gesamtpaket ausmacht. Durch die Zusammenarbeit verschiedenster Sensoren am Fahrzeug, ist es möglich, die Fahrt sicher zu gestalten.

Im ersten Kapitel dieser Arbeit wurde die Funktionsweise von autonomen Fahrzeugen allgemein erklärt und die Autonomiestufen, die angeben, auf welchem Level ein Fahrzeug autonom unterwegs ist, genauer erläutert.

Während des Verfassens der Arbeit bin ich auch auf einige interessante Projekte gestoßen, die die Entwicklung von autonomem Fahren vorantreiben möchten. Seien es Tech-Giganten aus dem Silicon Valley oder die Verkehrsbetriebe der Stadt Wien, sie alle wollen unsere Zukunft durch Einführung der autonomen Mobilität verändern.


% 20180120T2131 Leonard Michlmayr

%% Einige Filter für die Einträge im Literaturverzeichnis
\defbibfilter{online}{( type=online or subtype=online )}
\defbibfilter{interview}{type=interview or subtype=interview}
\defbibfilter{onlinetext}{( type=online or subtype=online and not ( type=video
  or type=audio ) )}
\defbibfilter{offline}{not ( type=online or subtype=online )}
\defbibfilter{print}{not ( type=online or subtype=online or type=video or
  type=audio or type=interview or subtype=interview )}
\defbibfilter{offlinevideo}{type=video and not subtype=online}
\defbibfilter{offlineaudio}{type=video and not subtype=online}
\defbibfilter{nurAusSekundaerliteratur}{category=quotee and not category=primary}
\defbibfilter{nichtNurAusSekundaerliteratur}{category=quoter or category=primary}

\printbibheading[heading=bibintoc,title=Literaturverzeichnis]\label{Lit}
\printshorthands[heading=subbibintoc]
\printbibliography[heading=subbibintoc,title={Print-Quellen},filter=print,filter=nichtNurAusSekundaerliteratur]
\printbibliography[heading=subbibintoc,title={Audio-Quellen},filter=offlineaudio,filter=nichtNurAusSekundaerliteratur]
\printbibliography[heading=subbibintoc,title={Video-Quellen},filter=offlinevideo,filter=nichtNurAusSekundaerliteratur]
\printbibliography[heading=subbibintoc,title={Internet-Quellen},filter=online,filter=nichtNurAusSekundaerliteratur]
\printbibliography[heading=subbibintoc,title={Sekundärzitate},filter=nurAusSekundaerliteratur]
\printbibliography[heading=subbibintoc,title={Interviews},filter=interview]

\listoffigures
\begingroup
\let\clearpage\relax
\listoftables
\endgroup
\appendix
\chapter{Akronyme und Abkürzungen}

\begin{acronym}[XXXXXX]
  \acro{BLMR}{Back Lane Marking Registry}
  \acro{GPS}{Global Positioning System}
  \acro{InGaAs}{Indiumgalliumarsenid}
  \acro{Lidar}{Light detection and ranging}
  \acro{NHTSA}{National Highway Traffic Safety Administration}
  \acro{Radar}{Radio detection and ranging}
  \acro{SAE}{SAE International}
\end{acronym}


\backmatter

\pdfbookmark[0]{Erklärungen}{erkl}
\addchap{Erklärungen}
\section*{Selbstständigkeitserklärung}
\thispagestyle{plain}
Ich erkläre, dass ich diese vorwissenschaftliche Arbeit eigenständig
angefertigt und nur die im Literaturverzeichnis angeführten Quellen und
Hilfsmittel benutzt habe.

\vspace{2cm}\noindent Wien, \today


\vspace{2cm}\noindent\makeatletter\@AutorIn\makeatother

\vspace{2cm}\noindent

\section*{Informatikschwerpunkt}

Die vorliegende Arbeit erfüllt die Kriterien zur Abbildung des
Informatikschwerpunktes an der De La Salle Schule Strebersdorf, AHS.

\textbf{Begründung:} Die Arbeit wurde in \LaTeX{} mit entscheidenden
Kenntnissen zum Quelltext verfasst.\vspace{.5\baselineskip}

\noindent\textit{Geprüft am 11.\,Dezember 2018 durch Mag.\ Leonard Michlmayr und Prof.\ Mag.\ Kurt~Rauch.}


\end{document}
