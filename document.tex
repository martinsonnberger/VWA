% Vorlage für die VWA. Version 20180117 (c) Leonard Michlmayr

% TODO: Wähle den für deine Arbeit die passenden Optionen!
\documentclass[DLS,
	inreferencehack,
	ohneVgl=true,
	ohneS=false,
	noscauthor,
	rundeauslassung=false,
	bookstyle=false,
	widowlines=3]{vwa}

%% Chapter Style
\usepackage{graphicx}
\usepackage{titlesec}

\newcommand*\HUGE{\Huge}
\newcommand*\chapnamefont{\normalfont\LARGE\MakeUppercase}
\newcommand*\chapnumfont{\normalfont\HUGE}
\newcommand*\chaptitlefont{\normalfont\HUGE\bfseries}

\newlength\beforechapskip
\newlength\midchapskip
\setlength\midchapskip{\paperwidth}
\addtolength\midchapskip{-\textwidth}
\addtolength\midchapskip{-\oddsidemargin}
\addtolength\midchapskip{-1in}
\setlength\beforechapskip{18mm}

\titleformat{\chapter}[display]
  {\normalfont\filleft}
  {{\chapnamefont\chaptertitlename}%
    \makebox[0pt][l]{\hspace{.8em}%
      \resizebox{!}{\beforechapskip}{\chapnumfont\thechapter}%
      \hspace{.8em}%
      \rule{\midchapskip}{\beforechapskip}%
    }%
  }%
  {25pt}
  {\chaptitlefont}
\titlespacing*{\chapter}
  {0pt}{40pt}{40pt}

%% Abkürzungen
\newcommand{\zB}{z.\,B. }
\newcommand{\ua}{u.\,a. }

%% Trick texlipse to use biber instead of bibtex
\iffalse
\usepackage[error,backend=biber]{biblatex}
\fi
%%

\usepackage{textcomp}
\usepackage[output-decimal-marker={,}]{siunitx}
\usepackage{tabularx}
\usepackage{booktabs}
\usepackage{array}
\newcolumntype{P}[1]{>{\centering\arraybackslash}p{#1}}

% TODO: Link visit date aktualisieren
\addbibresource{quellen.bib}
\graphicspath{{img/}}

\Autor{Martin Sonnberger}
\Klasse{8C}
\Betreuer{Mag. Leonard Michlmayr}
\Thema{Positionsbestimmung beim autonomen Fahren}

% TODO: erst bei der letzten Version das Abgabedatum anführen
% \Abgabedatum{}

\begin{document}

% Am Anfang keine Seitennummern
\frontmatter

% PDF-Lesezeichen für die Titelseite
\pdfbookmark[0]{Titelseite}{titlepage}
% Titelseite
\maketitle

% TODO: Abstract einfügen
% \addchap{Abstract}

In dieser vorwissenschaftlichen Arbeit wird der Prozess des autonomen Fahrens genauer erläutert. Insbesondere wird die Frage beantwortet, welche Methoden angewandt werden, um die Position eines autonomen Fahrzeugs präzise bestimmen zu können.
\bigskip

Im ersten Teil der Arbeit wird der Begriff des autonomen Fahrens und die verschiedenen Stufen, in die die Autonomie eines Fahrzeugs eingeteilt werden können, erklärt. Außerdem werden die im Fahrzeug verbauten Sensoren sowie der Prozess des autonomen Fahrens behandelt. Fortgesetzt wird die Arbeit mit den verschiedenen Methoden der Lokalisierung, abschließend wird der aktuelle Entwicklungsstand, sowohl von Produkten zur Positionsbestimmung als auch von Projekten zum autonomen Fahren, dargelegt.
\bigskip

Nach intensiver Auseinandersetzung mit dem Thema habe ich die Erkenntnis gewonnen, dass noch keine der in dieser Arbeit behandelten Technologien eine endgültige Lösung für das Problem der notwendigen genauen Positionsbestimmung darstellt. Letztendlich müssen die Fahrzeughersteller wohl auf eine Kombination verschiedener Systeme zurückgreifen um den hohen Anforderungen gerecht zu werden.


% TODO: eventuell Vorwort einfügen
% \include{vorwort}

% Das Inhaltsverzeichnis soll ein PDF-Lesezeichen aber keinen Eintrag im
% Inhaltsverzeichnis haben.
\cleardoublepage\pdfbookmark[0]{\contentsname}{toc}
% Inhaltsverzeichnis
\tableofcontents

% Hier geht es los.
\mainmatter

\chapter{Autonomes Fahren --- Beschreibung allgemein}
\section{Begriffsdefintion}
Um den Begriff \enq{Autonomes Fahren} definieren zu können, muss in der Zeit ein Sprung zurück gemacht werden, nämlich zur Erfindung des Automobils im Jahr 1886. Der Begriff \enq{Automobil} setzt sich aus dem griechischen \textit{autòs} (\enq{selbst}) und dem lateinischen \textit{mobilis} (\enq{beweglich}) zusammen, bedeutet also \enq{Selbstbewegliche}. Damals sollte dadurch, die von nun an bestehende Unabhängigkeit von Pferden verdeutlicht werden. Genauer betrachtet fällt auf, dass durch das Wegfallen von Kutschpferden die Autonomie in gewisser Weise verloren ging. Durch jahrelanges Training konnten sich die Pferde in Maßen selbst fortbewegen, indem sie beispielsweise einen nicht mehr fahrtüchtigen Kutscher alleine nachhause brachten.

Durch die Einführung von zahlreichen Fahrerassistenzsystemen und letzendlich des autonomen Fahrens wird dem Automobil seine ursprünglich gedachte Autonomität nicht nur wieder zurückgegeben, sondern auch um ein Vielfaches erweitert.

\zit{Definition}{Autonomes Fahren bedeutet das selbständige, zielgerichtete Fahren eines Fahrzeugs im realen Verkehr, ohne Eingriff des Fahrers.}

\section{Automatisierungsgrade}
Der Grad der Automatisierung kann in verschiedene Stufen eingeteilt werden.



% 20180120T2131 Leonard Michlmayr

%% Einige Filter für die Einträge im Literaturverzeichnis
\defbibfilter{online}{( type=online or subtype=online )}
\defbibfilter{interview}{type=interview or subtype=interview}
\defbibfilter{onlinetext}{( type=online or subtype=online and not ( type=video
  or type=audio ) )}
\defbibfilter{offline}{not ( type=online or subtype=online )}
\defbibfilter{print}{not ( type=online or subtype=online or type=video or
  type=audio or type=interview or subtype=interview )}
\defbibfilter{offlinevideo}{type=video and not subtype=online}
\defbibfilter{offlineaudio}{type=video and not subtype=online}
\defbibfilter{nurAusSekundaerliteratur}{category=quotee and not category=primary}
\defbibfilter{nichtNurAusSekundaerliteratur}{category=quoter or category=primary}

\printbibheading[heading=bibintoc,title=Literaturverzeichnis]\label{Lit}
\printshorthands[heading=subbibintoc]
\printbibliography[heading=subbibintoc,title={Print-Quellen},filter=print,filter=nichtNurAusSekundaerliteratur]
\printbibliography[heading=subbibintoc,title={Audio-Quellen},filter=offlineaudio,filter=nichtNurAusSekundaerliteratur]
\printbibliography[heading=subbibintoc,title={Video-Quellen},filter=offlinevideo,filter=nichtNurAusSekundaerliteratur]
\printbibliography[heading=subbibintoc,title={Internet-Quellen},filter=online,filter=nichtNurAusSekundaerliteratur]
\printbibliography[heading=subbibintoc,title={Sekundärzitate},filter=nurAusSekundaerliteratur]
\printbibliography[heading=subbibintoc,title={Interviews},filter=interview]

\listoffigures
\listoftables

\backmatter

%\pdfbookmark[0]{Erklärungen}{erkl}
\addchap{Erklärungen}
\section*{Selbstständigkeitserklärung}
\thispagestyle{plain}
Ich erkläre, dass ich diese vorwissenschaftliche Arbeit eigenständig
angefertigt und nur die im Literaturverzeichnis angeführten Quellen und
Hilfsmittel benutzt habe.

\vspace{2cm}\noindent Wien, \today

% TODO: Erkläre dich selbstständig selbstständig!
\vspace{2cm}\noindent\makeatletter\@AutorIn\makeatother

\vspace{2cm}\noindent

\section*{Informatikschwerpunkt}

Die vorliegende Arbeit erfüllt die Kriterien zur Abbildung des
Informatikschwerpunktes an der De La Salle Schule Strebersdorf, AHS.

\textbf{Begründung:} Die Arbeit wurde in \LaTeX{} mit entscheidenden
Kenntnissen zum Quelltext verfasst.\vspace{.5\baselineskip}

\noindent\textit{Geprüft am \ldots durch Mag. Rainer Zufall und Mag.
Ernst Haft}

\end{document}
