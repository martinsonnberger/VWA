\chapter{Zusammenfassung und Fazit}

Diese Arbeit hat die Frage behandelt, welche Methoden zur Positionsbestimmung bei autonomen Fahrzeugen angewandt werden.

Durch Auseinandersetzung mit dem Thema bin ich zur Erkenntnis gekommen, dass die Hersteller autonomer Fahrzeuge derzeit noch auf viele verschiedene Systeme zur Lokalisierung der Fahrzeuge setzen. Noch ist keine der Technologien ausgereift genug, um massentaugliche und erschwingliche autonome Fahrzeuge für den Alltag herstellen zu können. \acs{Lidar} arbeitet auf einem hohen Genauigkeitslevel, ist jedoch noch zu teuer und auch nicht platzsparend. \ac{GPS} hingegen ist kostengünstig, allerdings für eine präzise Positionsbestimmung nicht ausreichend genau.

Letztendlich ist es eine Kombination der Systeme, die ein gutes Gesamtpaket ausmacht. Durch die Zusammenarbeit verschiedenster Sensoren am Fahrzeug, ist es möglich, die Fahrt sicher zu gestalten.

Im ersten Kapitel dieser Arbeit wurde die Funktionsweise von autonomen Fahrzeugen allgemein erklärt und die Autonomiestufen, die angeben, auf welchem Level ein Fahrzeug autonom unterwegs ist, genauer erläutert.

Während des Verfassens der Arbeit bin ich auch auf einige interessante Projekte gestoßen, die die Entwicklung von autonomem Fahren vorantreiben möchten. Seien es Tech-Giganten aus dem Silicon Valley oder die Verkehrsbetriebe der Stadt Wien, sie alle wollen unsere Zukunft durch Einführung der autonomen Mobilität verändern.
