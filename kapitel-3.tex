\chapter{Aktueller Entwicklungsstand}

Dieses Kapitel soll einen Überblick über derzeitige Projekte schaffen, die es sich zur Aufgabe gemacht haben, autonomes Fahren weltweit auf die Straßen zu bringen.

\section{Waymo}

Waymo ist ein Schwesterunternehmen von Google und führt das im Jahr 2009 gegründete \textit{Google Driverless Car}-Projekt fort. 2015 fand die erste, voll autonome Fahrt auf öffentlichen Straßen in Austin statt. Im Oktober 2018 erreichte Waymo den Meilenstein von 10 Millionen zurückgelegten Meilen (ca. 16 Millionen Kilometer) in ihrer Flotte von autonomen Fahrzeugen. Noch 2018 will das Unternehmen ein voll autonomes, kommerzielles Taxi-Service in Phoenix starten, wofür Waymo auch einen Vertrag über \num{20000} Jaguar I-PACE Modelle abgeschlossen hat, um in Zunkunft nicht nur mit autonomen, sondern auch mit elektrischen Fahrzeugen, unterwegs sein zu können. \vgls{waymo-taxiservice}{waymo} Bereits seit 2017 läuft das sogenannte \textit{Early rider program}, an dem ausgewählte Bewohner des Ballungsraums von Phoenix teilnehmen können, um kostenlose Taxifahrten zu erhalten. Das Feedback der Teilnehmer trägt zur Weiterentwicklung der autonomen Fahrzeuge bei.
\vgl{waymo}


\section{Tesla}

Tesla ist, ebenso wie Waymo, ein im Silicon Valley ansässiges Unternehmen mit folgendem selbst ernannten Ziel: \zit{tesla-about}{Die Beschleunigung des Übergangs zu nachhaltiger Energie.} 
